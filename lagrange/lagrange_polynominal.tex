\documentclass[oneside,final,14pt]{extreport}
\usepackage[utf8]{inputenc}
\usepackage[english,russian]{babel}
\usepackage{cmap}
\usepackage{mathtext}
\usepackage{amsmath, amsfonts, amssymb} 
\sloppy
\begin{document}
\title{Интерполяционный полином Лагранжа}
%\date{}
\author{Виктор Филинков}
\maketitle

\chapter*{Измерения}
\begin{table}[pht]
	\centering
	\begin{tabular}{|r | c | c|}
		\hline \textbf{\#} &  \textbf{Нагрузка} & \textbf{Пульс} \\ \hline
		0 & 0	&	68	\\ \hline
		1 & 10	&	76	\\ \hline
		2 & 25	&	88	\\ \hline
		3 & 40	&	94	\\ \hline
	\end{tabular}
	%\caption{Измерения}
	\label{measurements}
\end{table}

\chapter*{Поиск многочлена}
\section*{Формулы}

\[
	L(x)=\sum_{i=0}^{n} y_i \prod_{j = 0, j \ne i} \frac{x-x_j}{x_i-x_j}
\]

\section*{Вычисления}
\begin{multline*}
%\\
	L(x)= \\
	68 \frac{(x-10)(x-25)(x-40)}{(-10) (-25) (-40)}  + 
	76 \frac{x(x-25)(x-40)}{10(10-25)(10-40)} \\ + \\
	88 \frac{x(x-10)(x-40)}{25(25-10)(25-40)} +
	94 \frac{x(x-10)(x-25)}{40(40-10)(40-25)} \\ = \\
	68 \frac{x^3-75x^2+1650x-10000}{-1000} + % числители получены при помощи maxima
	76 \frac{x^3-65x^2+1000x}{4500} \\ + \\
	88 \frac{x^3-50x^2+400x}{-5625} +
	94 \frac{x^3-35x^2+250x}{18000} \\
\end{multline*}


\end{document}